\subsection{Company Structure & Business Model}
The two holding companies for automobiles production, Fiat and Chrysler Groups, became a multi-national organization when they merged in early 2014 \cite{fca_history}. With business coverage spread over 140 countries, an employment of nearly 236,000 people, this strategic fusion lead FCA (i.e. "Fiat Chrysler Automobiles") to be one of the biggest automotive manufacturers in the whole planet. The FCA group now has control over the whole chain of production and services, including design, engineering, manufacturing, vehicles and spare components sales, and provision of a multitude of services \cite{fca_overview}. The business operativity is active worldwide through the 87 research and development centres, 159 manufacturing facilities, as well as a multitude of dealers and distributors. The production of technology has been driven by the sub-brand Magneti Marelli, which has provided FCA with the opportunity of having consistent and innovative development of vehicle components and low dependency from third party supplies \cite{magneti_marelli}. Other businesses within the FCA group further contribute to the complete control of the vehicle production, including Comau for the development of production systems and sub-brand for iron and castings, Teksid \cite{fca_overview}.\n 


Mopar is also an integral part of the FCA group, previously being a part of Chrysler Corporation. Mopar, whose name stands for "MOtor PARts" finds its target on  offering quality accessories and aftermarket parts for FCA vehicles as well as maintenance services for owners \cite{fca_mopar}. Having this sub-brand within the FCA group ensures a compatibility and direct connection to the original product that no other aftermarket parts company can provide \cite{fca_mopar}. Additionally, the services provided to owners, such as vehicle protection plans, roadside assistance and express care should ensure that the vehicle is safe and in perfect order. These efforts are signs that FCA is striving to create standardization, quality and services that are close to the customer and make use of FCA's high presence in markets worldwide. 
 
Currently, FCA delivers vehicles to customers through the brands Abarth, Alfa Romeo, Chrysler, Dodge, Fiat, Fiat Professional, Jeep, Lancia, Ram and Maserati \cite{fca_overview}. Within this portfolio of brands, FCA can offer a wide variety of vehicles in a number of different segments. This includes passenger cars ranging from the A-segment (i.e. city cars), where FCA has a history of successes, such as the Fiat Panda and the Fiat 500, all the way up to luxury F-segment cars, such as the Maserati Quattroporte, but also sport utility vagons where Jeep is a well known player, multi-purpose vehicles with the Dodge Caravan as an example, the pickup-truck dedicated brand Ram, labour vehicles such as Fiat Fiorino and Alfa Romeo or Maserati when it comes down to sporty vocations.
The choice to maintain a high level of diversity in the production regime ensures a better penetration in different demographic and geographic zones, as the historical success has always proved \cite{fca_marketing_strategy}.


\subsection{Strategic growth and vision}
Is not a secret that the two main groups that compose FCA have had hard times during the decade 2000-2010.
Since then, FCA operational directives were strategically focused upon a completely new, fully (economically) sustainable approach. Most of this changes were applied by the former company CEO Sergio Marchionne, who has restrengthened the Fiat brand by merging with Chrysler, optimizing resources and processes, exiting low-revenue markets and rearranging the personnel structure, improving teaming and cooperation \cite{fca_marketing_strategy}. Financially speaking, the results of such complete overhaul have been impressive: FCA was transformed from a almost zero growth company into one of the biggest giants in the automotive industry in less than ten years, leading to a staggering increase in company revenues: in 2017 alone, net revenues almost doubled the preceding year. \cite{fca_marchionne_fortune}\cite{statista_fca_net_revenues}.

The report of the business plan from June 2018 gives an indication of the strategy for the company the following four years up until 2022. Presented in the report, increased efficiency is the main strategic focus for the company \cite{fca_financial_overview}. The key that allowed to massively raise profits in the past years and that will allow to further increment them is the so called "World Class Manufacturing" (WCM), a framework of methodologies inspired by the Japanese lean production philosophy \cite{fca_wcm}. Within the optics of the WCM, a working team face every decision is going to take (maintanance, logistic, quality assurance, safety, management, ...) on the basis of their economical incidence. The team activity is oriented toward the accomplishment of small, focused projects (Kaizen) whose objectives are "zero defects, zero faults, zero incidents, zero remaining stocks": in other words, the final target is the drastic reduction of the production plant costs.

La metodologia TPM (Total Productive Maintenance), le logiche Lean Manufacturing (Produzione snella) ed il Total Quality Management, sono stati integrati, e sono la base del W.C.M. (World Class Manufacturing) che, pur basandosi su moltissimi concetti del TPM / TQM si differenzia da questi in quanto alla base della scelta delle strategie e degli impianti "critici" (in gergo detti " da aggredire") vi è il cosiddetto Cost Deployment. Ciò significa che il gruppo di lavoro affronta le problematiche, siano esse manutentive, logistiche, qualitative, di sicurezza, organizzative, di organizzazione del posto di lavoro, sulla base della loro incidenza economica. Le attività di tutti i team sono orientate alla realizzazione di progetti (Kaizen) i cui obiettivi sono: zero difetti, zero guasti, zero incidenti e zero scorte, finalizzate ad una generale riduzione dei costi dello stabilimento.

Questa nuova metodologia si sta diffondendo laddove si vuole controllare e ridurre i costi produttivi in maniera sistemica e con metodi riferibili ed oggettivabili.

La strategia del W.C.M. è stata sviluppata dal gruppo FIAT (oggi FCA), del quale si sta rivelando una delle ragioni della recente svolta positiva.



The business plan also introduces a forecast regarding the electrification of vehicles within the FCA group in the future. In the last years, FCA has displayed skepticism regarding the electric cars. (Detroit news) Despite this, however, for the following years, FCA is going to significantly expand in the electrification of vehicles \cite{fca_financial_overview}. All-electric vehicles will be most prominent in the lineup of upcoming Maserati and Jeep vehicles, with a total of 10 battery electric vehicles waiting to be introduced. 



\bibliographystyle{plain}
\bibliography{references/fca_BM_G_V.bib}