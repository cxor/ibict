\subsection{Company Structure & Business Model}
The two holding companies for automobile production, Fiat Group and Chrysler Group, became a multi-national organization when they merged in early 2014 \cite{fca_history}. With business coverage spread over 140 countries, an employment of nearly 236,000 people, this strategic fusion lead FCA to be one of the biggest automotive manufacturers in the whole planet. The FCA group now has control over the whole chain of production and services, including design, engineering, manufacturing, vehicles and spare components sellings, and provision of a multitude of services \cite{fca_overview}. The business operativity is active worldwide through the 87 research and development centres, 159 manufacturing facilities, as well as a multitude of dealers and distributors. The production of technology has been driven by the sub-brand Magneti Marelli, which has provided FCA with the opportunity of having consistent and innovative development of vehicle components and low dependency of other production companies \cite{magneti_marelli}. Other businesses within the FCA group further contribute to the complete control of the vehicle production, including Comau for the development of production systems and sub-brand for iron and castings, Teksid \cite{fca_overview}.\n 

Mopar is also an integral part of the FCA group, previously being a part of the Chrysler Corporation. Mopar, for motor parts, is there for owners of FCA vehicles to create a peace of mind, by offering quality accessories and aftermarket parts for FCA vehicles as well as services for owners \cite{fca_mopar}. Having this sub-brand within the FCA group shall ensure a compatibility and direct connection to the original product that no other aftermarket parts company can provide \cite{fca_mopar}. Additionally, the services provided to owners, such as vehicle protection plans, roadside assistance and express care should ensure that the vehicle is safe and in perfect order. These efforts are signs that FCA is striving to create standardisations, quality and services that are close to the customer and make use of FCAs high presence in markets worldwide. 
 
Currently, FCA delivers vehicles to customers through the brands Abarth, Alfa Romeo, Chrysler, Dodge, Fiat, Fiat Professional, Jeep, Lancia, Ram and Maserati \cite{fca_overview}. Within this portfolio of brands, FCA can offer a wide variety of vehicles in a number of different segments. This includes passenger cars ranging from the A-segment, where FCA has a history of successes, such as the Fiat Panda and the Fiat 500, all the way up to luxury F-segment cars, such as the Maserati Quattroporte. 
Further, FCA is represented in the full range within the remainder of car segments. This includes SUVs where Jeep is a big player, MPVs with the Dodge Caravan as an example, the pickup-truck dedicated brand Ram, LVCs with Fiat Fiorino as one instance and also prominent players such as Alfa Romeo and Maserati in the sports car segment. With a wide range of sub-brands in the FCA group, the company can cover a variety of segments in different markets around the world. The diverse portfolio ensures that FCA is present for consumers in a range of demographic, geographic and psychographic segments. Having a large number of brands in its product portfolio makes it beneficial for FCA in being able to compete in a variety of segments, its ability to penetrate emerging markets and as far as economies of scale are concerned(Marketing91).


\subsection{Strategic growth and vision}
FCA has had a strategic approach that is focused on economic sustainability during the last decade. Much of this has been due to the impact of Sergio Marchionne as a CEO of the company, which has restrengthened the Fiat brand by merging with Chrysler, optimising resources and processes, exiting loss markets and decreased number of employees among other measures(Marketing91). This has transformed a struggling company to one of the big giants in the automotive industry, leading to an increase in company revenue during Marchionne’s time as a CEO \cite{fca_marchionne_fortune}\cite{statista_fca_net_revenues}.

The report of the business plan from June 2018 gives an indication of the strategy for the company the following four years up until 2022. Presented in the report, increased efficiencies are continuing to be a strategic focus for the company(FCA report). This includes both in manufacturing through quality and process improvements, as well as increased architecture sharing between cars for higher purchasing efficiency. Further, continued focus is in the so-called world-class manufacturing, a process inspired by the Japanese lean production philosophy(World class manufacturing). The plan also presents more white-space products in highly profitable segments in the market. Focus on UVs for the coming four years, since they want to capitalise on expected high customer preference for UVs in all markets. Mainly, this concerns Jeep, Alfa Romeo and Maserati products. 

The business plan also introduces a forecast regarding the electrification of vehicles within the FCA group in the future. In the last years, FCA has displayed scepticism regarding the electric cars. (Detroit news) Despite this, however, for the following years, FCA is going to significantly expand in the electrification of vehicles (FCA report). All-electric vehicles will be most prominent in the lineup of upcoming Maserati and Jeep vehicles, with a total of 10 battery electric vehicles waiting to be introduced. 



\bibliographystyle{plain}
\bibliography{references/fca_BM_G_V.bib}