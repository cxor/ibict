\subsection{Environmental Sustainability}
In the beginning, the reason that drove Martin Eberhard and Marc Tarpenning to create Tesla was the recall and destruction of all the electric cars by General Motors\cite{muskGM} in 2003. This, united with Tesla's mission to \emph{accelerate the world’s transition to sustainable energy through increasingly affordable electric vehicles and energy products}\cite{aboutTesla}.

For this to be effective, it starts from the factories. Tesla, as a matter of facts, has built the so called Giga-factories (namely Gigafactory 1 and Gigafactory 2), a facility designed to be fully powered using renewable energy. These two factories are located one near Sparks, Nevada and one in Buffalo, New York, and are used respectively to produce lithium-ion batteries and solar panels.
Another one of these factories, the Gigafactory 3, will be built in Shangai, in China. 
Tesla is also active in the production of solar panels with its subsidiary SolarCity.

All the vehicles produced by Tesla are powered by electric energy, reducing greatly the emissions produced in respect to a fossil fuel-powered vehicle. This, however, makes a higher number of batteries necessary. 
One of the problems with the increase of electric cars in general, not only Tesla, is the environmental impact of the batteries, both for their production and their disposal. 
The emissions produced by an electric vehicle are mainly composed by two factors:
\begin{itemize}
    \item the greenhouse gas that is generated for the production of the energy needed to recharge them;
    \item the environmental pollution of the extraction of the minerals needed for the creation of the energy cell\cite{scheele2016cobalt}.
\end{itemize}
 A study found out that Tesla vehicles have a CO$_2$ payback in about 3 years \cite{electricStudy}. This of course depends on how much the vehicle is used. 

The environmental impact of Tesla batteries is also reduced by the fact that all of them are currently recycled by partner companies, and will be recycled directly at the gigafactories in the future, to reuse the materials in the production of new ones, as the CTO Straubel said \cite{recycleBattery}.

\subsection{Maintenance}

The motor of a Tesla car has less moving parts than the one of a fossil fuel powered one, thus greatly improving its duration and the possible malfunctioning. As a matter of facts, a picture of a Tesla motor which lasted for over one million miles in a test was published on Twitter, and it looked like it just came out of the factory\cite{tesla1mMiles}.

A Tesla car has a 4 years (or 50000 miles, whichever comes first) warranty, which covers repairs or replacements needed to correct defects in the materials or any part manufactured or supplied by Tesla which come with normal use. On top of that, the battery and the drive unit of the car are covered by an 8-year warranty. This one covers every replacement or repair done to correct a malfunctioning or defective piece, also covering the damage to the vehicle from a battery fire if that ever occurs, even if it is resulting from a driver error\cite{teslaWarranty}. 

Tesla also offers some maintenance plans. They cover the annual service inspections for 3 or 4 years. This can be purchased when purchasing the car but, unfortunately, not worldwide. 

For any problem with their car, accidental or not, Tesla offers service centers. These centers are all over the world and their number is continuously expanding year by year. A service appointment can be scheduled directly in the car with the on-board computer, It is not always necessary to drive to these service centers, this is because, especially in the USA, the mobile service exists. This consists in an expanding fleet of expert technicians driving even to the home of the customers in order to solve their problem when possible.
