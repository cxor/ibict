\subsection{Environmental Sustainability}
FCA has always been aware of the fact that petrol engines cars are one, among with others, of the main cause of environmental pollution. This is one of the reasons that led the company to start looking for a new, better and less harmful way to power the engines of their vehicles.\\
After investing lots of money and resources to achieve this goal, FCA started developing engines based on LPG gas and after more than 30 years from the first LPG model launched on the market, FCA is still one of the biggest producer of natural gas vehicles in the world\cite{FCA_sustainability}.

Of course this is not the ultimate solution to reduce global environmental pollution, but it certainly is a starting point. FCA did not stop there, indeed it is still looking for better and cleaner ways to power their vehicles and to produce the materials needed for the production. As a matter of fact they are experimenting different approaches that can both suit the needs of the company and that are friendly to the environment. This is why they are about to launch on the market 5 new completely electric Maserati models\cite{Maserati_electric}.

Talking about electric cars, people may think that are the perfect solution to the pollution problem, but is this true?
The idea is having vehicles that do not pollute while they are driven, but how can a company as FCA make it possible and profitable?\\
The main problem is that most of the electricity produced is derived from non-renewable sources and that therefore are equally dangerous. A solution should be to find a way to support both factories and vehicles by using clean produced energy. Thus, FCA and the other companies should first find a solution for this problem, and then start planning the development and the release of fully electric vehicles.

Another equally critical aspect that must be considered, is that finding a solution to the environmental pollution problem doesn't concern only the companies but also the customers and every people around the world. For instance, the three most sold vehicles in the US are all pickup trucks\cite{US_best-selling-vehicle}. This is due to the lower price of the vehicle itself and especially of the gasoline. These aspects together with reliability and versatility, making customers much more likely to buy a pickup rather than a smaller, greener car.

In conclusion, FCA and the other companies must also find a way to convince people to buy more environmentally friendly cars, in order to really reduce and hopefully erase the pollution problem.

\subsection{Maintenance}

As a matter of fact, petrol cars have much more complex mechanisms compared to electric cars, in order to make the entire system work correctly. This also involves having more elements that need to be recycled or disposed of, both during the lifecycle of the vehicles and at the end of it.

The owners of a petrol engine cars have more maintenance costs compared to the ones of an electric vehicles. For instance, \emph{the average cost to operate an EV(Electric Vehicle) in the United States is \$485 per year, while the average for a gasoline-powered vehicle is \$1,117}\cite{maintenance_costs}. 

But what is FCA doing to meet the needs of its customers when it comes to vehicle maintenance? The answer is Mopar. 

With more than 80 years of experience, Mopar provides expert service, genuine parts and customer care specifically for the FCA vehicles around the globe\cite{maintenance_mopar}. 
The main goal of Mopar is to offer every possible part and accessory for each FCA model, from the SUVs to the smallest selling vehicles of the brand. Moreover, this elements are engineered by following factory-authorized specifications, and this guarantee that each one will perfectly fit inside the vehicles\cite{maintenance_mopar}.

In conclusion, \emph{Mopar is not only here to service cars, trucks, minivans and SUVs—it is a brand here to service people’s lives. This commitment defines who we are and fuels what we do. For this reason, everything we say and everything we do is aligned with the Mopar brand purpose:
ALL IN SERVICE OF THE PEOPLE WHO DRIVE US.}\cite{maintenance_mopar}

\section{Tesla View}

\subsection{Environmental Sustainability}
In the beginning, the reason that drove Martin Eberhard and Marc Tarpenning to create Tesla was the recall and destruction of all the electric cars by General Motors\cite{muskGM} in 2003. This, united with Tesla's mission to \emph{accelerate the world’s transition to sustainable energy through increasingly affordable electric vehicles and energy products}\cite{aboutTesla}.

For this to be effective, it starts from the factories. Tesla, as a matter of facts, has built the so called Giga-factories (namely Gigafactory 1 and Gigafactory 2), a facility designed to be fully powered using renewable energy. These two factories are located one near Sparks, Nevada and one in Buffalo, New York, and are used respectively to produce lithium-ion batteries and solar panels.
Another one of these factories, the Gigafactory 3, will be built in Shangai, in China. 
Tesla is also active in the production of solar panels with its subsidiary SolarCity.

All the vehicles produced by Tesla are powered by electric energy, reducing greatly the emissions produced in respect to a fossil fuel-powered vehicle. This, however, makes a higher number of batteries necessary. 
One of the problems with the increase of electric cars in general, not only Tesla, is the environmental impact of the batteries, both for their production and their disposal. 
The emissions produced by an electric vehicle are mainly composed by two factors:
\begin{itemize}
    \item the greenhouse gas that is generated for the production of the energy needed to recharge them;
    \item the environmental pollution of the extraction of the minerals needed for the creation of the energy cell\cite{scheele2016cobalt}.
\end{itemize}
 A study found out that Tesla vehicles have a CO$_2$ payback in about 3 years \cite{electricStudy}. This of course depends on how much the vehicle is used. 

The environmental impact of Tesla batteries is also reduced by the fact that all of them are currently recycled by partner companies, and will be recycled directly at the gigafactories in the future, to reuse the materials in the production of new ones, as the CTO Straubel said \cite{recycleBattery}.

\subsection{Maintenance}

The motor of a Tesla car has less moving parts than the one of a fossil fuel powered one, thus greatly improving its duration and the possible malfunctioning. As a matter of facts, a picture of a Tesla motor which lasted for over one million miles in a test was published on Twitter, and it looked like it just came out of the factory\cite{tesla1mMiles}.

A Tesla car has a 4 years (or 50000 miles, whichever comes first) warranty, which covers repairs or replacements needed to correct defects in the materials or any part manufactured or supplied by Tesla which come with normal use. On top of that, the battery and the drive unit of the car are covered by an 8-year warranty. This one covers every replacement or repair done to correct a malfunctioning or defective piece, also covering the damage to the vehicle from a battery fire if that ever occurs, even if it is resulting from a driver error\cite{teslaWarranty}. 

Tesla also offers some maintenance plans. They cover the annual service inspections for 3 or 4 years. This can be purchased when purchasing the car but, unfortunately, not worldwide. 

For any problem with their car, accidental or not, Tesla offers service centers. These centers are all over the world and their number is continuously expanding year by year. A service appointment can be scheduled directly in the car with the on-board computer, It is not always necessary to drive to these service centers, this is because, especially in the USA, the mobile service exists. This consists in an expanding fleet of expert technicians driving even to the home of the customers in order to solve their problem when possible.


\bibliographystyle{plain}
\bibliography{references/environment.bib}