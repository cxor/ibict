\section{Environmental Sustainability}
\subsection{FCA View}
FCA has always been aware of the fact that petrol engines cars are one of the main cause of environmental pollution. This is one of the reasons that led the company to start looking for a new, better and less harmful way to power the engines of their vehicles.\\
After investing lots of money and resources to achieve this goal, FCA started developing engines based on LPG gas and after more than 30 years from the first LPG model launched on the market, FCA is still one of the biggest producer of natural gas vehicles in the world\cite{FCA_sustainability}.\\
Even if this could not be a long-range solution, the importance of having a sustainable alternative to fuel vehicles has to be taken into account. Furthermore, FCA does not want to fully invest in not-mature technologies, but it tries to develop them in parallel with well-established ones.
Of course this is not the ultimate solution to reduce global environmental pollution, but it certainly is a starting point. FCA did not stop here, indeed it is still looking for a better and cleaner way to power their vehicles and to produce the materials needed for producing them. As a matter of fact they are experimenting different approaches that can both suit the needs of the company and that are friendly to the environment. This is why they are about to launch on the market 5 new completely electric Maserati models\cite{Maserati_electric}.

Talking about electric cars, people may think that are the perfect solution to the pollution problem, but is it?
The idea is having vehicles that do not pollute while they are driven, but how can a company as FCA make it possible?\\
The main problem is that most of the electricity produced is derived from non-renewable sources and that therefore are equally dangerous. A solution should be to find a way to support its factories and their vehicles by using clean produced energy.

Another important aspect that must be considered, is that the environmental pollution problem doesn't concern only the companies but also the customers. For instance, the three most sold vehicles in the US are all pickup trucks\cite{US_best-selling-vehicle}. This is due to the lower price of the vehicle itself and especially of the gasoline. 
FCA and the other companies has to find a way to also convince people to buy more environmentally friendly cars, in order to really reduce and hopefully erase the pollution problem.

\subsection{Tesla View}

In the beginning, the reason that drove Martin Eberhard and Marc Tarpenning to create Tesla was the recall and destruction of all the electric cars by General Motors\cite{muskGM} in 2003. This, united with Tesla's mission to \emph{accelerate the world’s transition to sustainable energy through increasingly affordable electric vehicles and energy products}\cite{aboutTesla}.

For this to be effective, it starts from the factories. Tesla, as a matter of facts, has built the so called Giga-factories (namely Gigafactory 1 and Gigafactory 2), facilities designed to be fully powered using renewable energy. These two factories are located one near Sparks, Nevada, and one in Buffalo, New York, and are used respectively to produce lithium-ion batteries and solar panels.
Another one of these factories, the Gigafactory 3, will be built in Shangai, in chinese soil. 
Tesla is also active in the production of solar panels with its subsidiary SolarCity.

All the vehicles produced by Tesla are powered by electric energy, reducing greatly the emissions produced in respect to a fossil fuel-powered vehicle. This, however, makes a higher number of batteries necessary. 
One of the problems with the increase of electric cars in general, not only Tesla, is the environmental impact of the batteries, both for their production and their disposal. 
The emissions produced by an electric vehicle are mainly composed by two factors:
\begin{itemize}
    \item the greenhouse gas that is generated for the production of the energy needed to recharge them;
    \item the environmental pollution of the extraction of the minerals needed for the creation of the energy cell\cite{scheele2016cobalt}.
\end{itemize}
 A study found out that Tesla vehicles have a CO$_2$ payback in about 3 years \cite{electricStudy}. This of course depends on how much the vehicle is used. 

The environmental impact of Tesla batteries is also reduced by the fact that all of them are currently recycled by partner companies, and will be recycled directly at the gigafactories in the future, to reuse the materials in the production of new ones, as the CTO Straubel said \cite{recycleBattery}.

\bibliographystyle{plain}
\bibliography{references/environment.bib}

