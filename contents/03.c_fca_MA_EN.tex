\subsection{Environmental Sustainability}
FCA has always been aware of the fact that petrol engines cars are one of main cause of environmental pollution, between many others. Being the ignition mechanisms the root of the problem itself, it's clear that it represents the starting point for a research of more ecologically sustainable solutions.
The first pivotal result came with, LPG gas based engines, almost 30 years ago, and since the launch of the first LPG vehicle, FCA is still one of the biggest producer of natural gas vehicles in the world \cite{FCA_sustainability}.
Nowadays LPG gas usage is considered more a short term solution, rather than a definitive one, and there are many reasons behind such perspective. The rationale is than LPG gas is not a renewable resource and, although not harmful as conventional petrol -as a combustible-, is still a not-towards-zero pollution method of ignition. As a result, FCA is experimenting different approaches, where the most significant one are electrical engines; although electricity can still questionably be considered not mature enough for a mass scale adoption, FCA has undergone some attempts to reach the market, as showed by the launch of five new completely electric Maserati models \cite{Maserati_electric}. The reasons behind considering electricity a final solution to the automotive pollution problem is underlying the fact that most of it is produced from non-renewable sources, which virtually doesn't add any benefit from an ecological standpoint.
Another equally critical aspect is that environmental pollution problem doesn't involve only the automotive companies, but also the customers: for instance, the three most sold vehicles in the US are all pickup trucks \cite{US_best-selling-vehicle}. This is due to the relatively low price of the vehicles in this category, and the almost unbeatable price of the required gasoline. These aspects, together with reliability and versatility concerns, are making customers much more prone to buy a pick-up rather than a smaller, greener car.
Given the complex nature of the problem, which intuitively has many fading shadows under its amplitude, FCA is trying to properly tackle the necessity of a sustainable transportation still pushing LPG technology for the upcoming years, but evaluating a diversified range of alternatives, heavily investing on the research of solutions which can fulfill the purpose, also if they appear immature at the present moment.
Conducting a successful business is not a mere process of introducing innovative ideas and technologies, but also to spot the right time when a certain solution has reached enough maturity to be effectively posed on the market. Exploiting this adequate level of maturity, consensus can be gained through mediatic momentum and therefore revenues will be automatically triggered. FCA knows it, and that's the actual motivation behind the conservative position about electric vehicles and their spreading as today.

\subsection{Maintenance}

A good brand does not only sells a quality product, but also has care of the services to provide in order to let the product itself gain an additional value.
In the automotive world, maintenance is often a crucial part of the whole picture, much more than many other sectors. FCA has a unique approach to maintenance services, bringing a centralized organism whose name is "Mopar". As already said, Mopar stands for "MOtor PARts", and has more than 80 years of history behind his back. Mopar provides assistance and replaceable parts for every FCA vehicle, and the only, specific commitment that this organism has is to furnish components that fit perfectly, guaranteeing availability, reliability and fast response times.
Having a unified entity that takes care of the plethora of maintenance services has several positive outcomes. First of all, knowledge and expertise can be managed in a more straightforward way, leading to working teams more homogeneously distributed in competences, and overall, able to inter-cooperate easily. Second, this choice fully integrates with the main whole business model that FCA conducts through the WCM, thus showing the benefits that the latter provides in the most effective way. Third, the marketing and technical sectors within Mopar can communicate with more practical and collaborative methods, since Mopar acts as it would be a detached company, managing its own personnel and working for its own profit \cite{maintenance_mopar}.
In our opinion, given these concerns, Mopar has few competitors that can reach the same professionality level in its own realm.