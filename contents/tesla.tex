\section{Tesla Motors}

\subsection{Business Model}
Tesla's founder and CEO Elon Musk launched the company with the mission of "accelerating the advent of sustainable transport, bringing electric cars en masse into the market as soon as possible." This sentence perfectly describes the big business model Tesla success.
Back in 2008, with the delivery of the first full electrical car named Roadster, Tesla entered the Electric Vehicles market, which probably was one of the less explored at that time.  
Nowadays, this company is able to exploit this market mainly because of its philosophy. In fact, Tesla Motors started developing its own cutting edge technology from the single battery cell to the engine gearbox, letting the company to rely on its own intellectual properties to build cars that shows incredible performances, not only in term of speed, but also in terms of battery degradation and range covered on a charge, showing to customers that this technology could be the future. 
It is important to note that, since all the Tesla’s intellectual properties are open source, every car manufacturer can further use and improve this technology.
Part of the Tesla’s business plan consists of the so-called “Masterplan” \cite{TeslaMasterplan}, used by the CEO Elon Musk to define the company’s strategy. As a first step, the company entered the market with high-end expensive cars to attract a relatively restricted circle of well-off customers (for example the Roadster was sold at 120'000\$ from the 2008 until the 2012). The second step consisted to release more affordable models, always respecting the high quality standard in terms of innovation; one prominent example is the Model S, available from 2012 and priced 65'000\$. In order to reach more customers, in 2017 the Model 3 was released starting at 35'000\$ and this represents the final step of the "Masterplan": being the price much affordable than preceeding models, Model 3 can clearly drastically broaden the market.
Along this selling strategy, the Business Model is composed by two other focal points.
In fact, the company provides a large and well developed network of charging point, the so-called “supercharging network”, which is very appreciated by the customers mainly because the company gives to them some free charges every year but also because it is ten time faster, i.e. about an hour, than charging the car at home.
Furthermore, one of the fundamental component of Tesla's business consists of the servicing facilities. Indeed, the presence of various centers all around the world permits the customers to request and buy repairs; moreover, newer models can be inspected remotely and specialized technicians can do maintenance directly at customers' house.

\subsection{Growth \& Vision}
Since Tesla's birth, its first goal was to incentivize the sustainable transport through a constant work aimed to create and refine the technology required to electric cars. Nonetheless, they are not only developing automobiles, indeed, in 2017 the company teased a new vehicle called Tesla Semi, which is an autonomous full electric truck with, as not fuel-powered vehicle, good recharge time and autonomy performances (around 800 kilometers). Furthermore, it boasts an energy consumption of less than 2 KWH per mile, which attracts the attention of many customers, interested in transport cost reduction. In fact, Semi, whose production should start in 2019, received pre-orders from various delivery companies.  
Moreover, the business of the company opened also to the house field with the presentation of the Tesla Powerwall \cite{Powerwall}, which is a big modular battery that can be installed at home and allows the storage of electrical energy produced by solar panels. Furthermore, on October 26th 2016, the CEO Musk announced a new product called “Tesla Roof Tiles” which consists of a root tile that can transfer the solar energy, through the means of electric energy, into an integrated Powerwall battery, providing it at any time.
The continuous growth of Tesla can be proved starting from its foundation in 2008, when the company was worth 14,7 million dollars. In fact, its value grew up to 111,9 million the following year with a 661\% surge. Successively, in 2010, Tesla went public with an initial public offering (IPO) of about 13,3 millions of shares at 17\$ each. Since then, the average value went up reaching about 360\$ per share at the beginning of December 2018, with a growth of around 2017\% in almost 8 years \cite{Growth}. 
In conclusion, the continuous development of electric batteries permits Tesla to broaden its views and try to become a good competitor in different markets, including, for example, the sustainable energy supply.

\subsection{Safety}

Safety and security have always been focal points of research within Tesla. Just to name few self-speaking results, every Tesla model released on the market has reached at least a 5 star rating out of 5 in the EuroNCap crash test, with the Model S, Model X and Model 3 outclassing the standard rating scale achieving an outstanding 5.4 out of 5 rating \cite{model_s_rating}.
By NHTSA standards (National Highway Traffic Safety Administration, the american agency monitoring safety measures in automotive trasportations) all Tesla cars set a new record in terms of safety -having the lowest probability of injury-, outclassing even recent Volvo cars, which historically have been leaders in the sector. For example, when compared to the Volvo S60, Tesla Model S preserved 63.5\% of driver residual space against the 7.8\% of the Volvo counterpart which, among other things, uses one of the most innovative solutions available (similar in characteristic to the one employed in the Apollo Lunar Lander) \cite{model_s_rating}.
Furthermore, electric vehicles are safer and more reliable by construction, since there is a drastic reduction in the number of moving and vibrating parts. Thus, their maintenance is greatly simplified, and also considerably cheaper \cite{aboutTesla}. Many concerns have been arose around the battery, since it's considered the most critical part in the entire vehicle structure: Tesla, in order to guarantee the highest level of safety possible, implements state-of-art technologies against overheating or burning danger. To better clarify how much the company cares in matter of safety and security, it's known that it has instructed firefighters to deal with a burning Tesla car \cite{tesla_training_fire}. Some innovations introduced in recent models allow to shift the car in autopilot mode whenever the board computer identifies some kind of danger, whether -for instance- the driver falls asleep or faints.
It's worth spending a brief commentary on the weaknesses of Tesla vehicles regarding advanced digital assets at their disposal. Cybersecurity, within the automotive world, is a pretty novel topic, and, although rather intuitively crucial, is not always taken seriously by vendors. On the contrary, Tesla was one of the first among all that has decided to adhere to severe guidelines of conduct in order to reduce the number of exploitable vulnerabilities; has also funded numerous campaigns to reward anyone who finds a new weakness in the digital systems adopted \cite{tesla_security_methods}.

\subsection{Self-Driving}

Tesla is arguably the absolute leader in self-driving technology. Its "Autopilot" toolset, now at the 9th version, is the most trusted self-driving system around \cite{tesla_trust}. Standing on official declarations, this system allows to reach a full, driverless, control of the vehicle, but, because of the limited time spent in testing, it is classified only as a "drive assistant", thus still relying on the presence of a physical driver \cite{tesla_full_self-driving}.
The nature of the driverless paradigm is quite a complex issue. As reported in \cite{tesla_awareness}, as of 2018 consumers are excited by the self-driving phenomenon, but also distrustful in its sudden adoption. Moreover, it shows that people generally prefer control over automation, and would be quite reluctant to get in an autonomous car \cite{tesla_awareness}. Although this situation seems to pose Tesla in a disadvantageous position, it is also true that is allowing the company to have the required time to deepen and extend the safety tests, thus gaining an increasing level of trust from the market, as well as social consensus.

\subsection{Environmental Sustainability}

Few people know that Tesla has been founded by pioneers such as Martin Eberhard and Marc Tarpenning because they wanted to avoid the disposal of electric cars produced by General Motors, back in 2003. \cite{muskGM} This simple "recycle" purpose was further transformed into a visionary target: \emph{"accelerate the world’s transition to sustainable energy through increasingly affordable electric vehicles and energy products"}, as showed by the official Tesla site \cite{aboutTesla}.

The project of environmental sustainability starts right from the factories, or Giga-factories as Tesla loves to call them. These facilities are designed to be fully powered using renewable energy; moreover, the production processes are highly optimized to harness the most modern energy conservation techniques, as well as less manual intervention as possible. In this way the overall result is the most effectively achievable with nowadays instruments. These two factories are located one near Sparks, Nevada and one in Buffalo, New York, and are used respectively to produce lithium-ion batteries and solar panels; another one will be built in Shangai, China. Tesla is also active in the production of renewable energy toolset, such as solar panels which the current production is demanded to the subsidiary SolarCity.

Considering that the vehicles produced by Tesla are only powered by electric energy, it is rather clear that the margin of emissions' reduction can be fairly huge; despite this fact, another substantial problem must be taken into account: battery production and disposal. Batteries are one of the most important element within the overall fully-electric architecture of Tesla cars, so it is not possible to simply find temporary workarounds that limit the problem, but do not solve it entirely. In fact, their construction requires high quantity of greenhouse gas -a non-renewable resource-, and more environmental pollution is generated by extraction of minerals needed to properly build energy-storing cells \cite{scheele2016cobalt}.

Tesla has currently many plans to tackle such issues, and some of them are under a testing period; today, the environmental impact of Tesla batteries is mitigated by recycling batteries through the means of partner companies, with prospects to do it within the Giga-factory alone by the end of a couple of years, as the CTO Straubel said \cite{recycleBattery}.

\subsection{Maintenance}

The motor of a Tesla car has less moving parts than the one of a fossil fuel powered one, thus greatly improving its duration and drastically reducing the possibility of a malfunction. Because of these unique characteristics, maintenance of Tesla vehicles is way easier and cheaper. A picture of a Tesla motor -which lasted for over one million miles in a test- was published on Twitter, and the exterior appearance is basically the same of brand new one\cite{tesla1mMiles}.

A Tesla car has a 4 years (or 50000 miles) warranty, which covers repairs or replacements needed to correct defects in the materials or any part manufactured or supplied by Tesla which come with normal use. Moreover, the battery and the drive unit of the car are covered by an extended 8-year warranty, also ensuring free of charge reparation from a battery fire -if it ever occurs!-, even in the case it's the result of a driver's error \cite{teslaWarranty}. 

Tesla also offers some optional maintenance plans, which cover the annual service inspections for 3 or 4 years. Although such assistance is not available worldwide, Tesla is pushing great efforts in order to broaden the covered areas. 

For any problem with their car, accidental or not, Tesla offers service centers placed all over the world, with their number in continuous expansion; it's also possible to book overhauls directly from the onboard computer. A more convenient alternative is also available: if required (especially in zones far away from the nearest center), an expert technician can be sent over to the customer's house and carry out the needed operations right on the place.