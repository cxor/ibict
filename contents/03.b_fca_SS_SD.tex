\subsection{Safety \& Security}

FCA takes safety very seriously, not only by a mere research point of view, but also when it comes down to liability. An outstanding example which justifies the latter assert comes from a recent event, where almost 5 millions vehicles have been subject to recall procedure due to a defect on the cruise control subsystem \cite{fca_5million_recall}. Although the issue can be considered of minor impact (since it affects only vehicles with more than 200'000 km, and the problem itself can occur in 'extremely rare circumstances', as stated by FCA itself), FCA opted to a full recall, even if it would imply significant expenses. Engineering issues can always happen, but these are the circumstances that highlight the reliability of a company, and where a quality service comes into play.
Safety, security, and in general terms, reliability are such primary concerns of FCA that a constant research is fullfilled in order to provide the bleeding-edge technology against threats or damages. Notorius recent examples are the Rear Cross Traffic Assist (RCTS) \cite{fca_rear_cross_traffic_assist}, which is helpful in blind driveways, the ArcelorMittal's latest Fortiform® alloy used in vehicles floors, which drastically reduces the costs, but also ensures higher torsion performances and reduces consumptions, due to the overall more lightweight structure \cite{fca_fortiform}.
Is also worth consider that safety and security are also tested by standard procedures, like crash tests, and lately FCA has exceedeed the 20'000 crash test from 1961 to today \cite{fca_20thousand_crash_tests}. This milestone represents more than 50 years of activity where FCA has joined a diverse range of technical skills and standardized methods in order to achieve safety level thant were once unthinkable. The excellence of results is also confirmed by the numerous quality awards conferred by independent third-party organizations, such as the Chrysler Pacifica recognition as "Top Safety Pick" by IIHS \cite{fca_pacifica_iihs}, which is one of the latest and most significant achievements.


\subsection{Self-driving technologies}

FCA started a strategic partnership with the self-driving car service Waymo, belonging under the Alphabet umbrella, since May 2016. The CEO at the time Sergio Marchionne commented: “Strategic partnerships, such as the one we have with Waymo, will help to drive innovative technology to the forefront. [...] FCA is committed to bringing self-driving technology to our customers in a manner that is safe, efficient and realistic”.
Those words entirely reflect the approach that FCA maintains to bring fresh new, qualitative technology into the automotive market.
A first achievement in this sector has been the production of 62,000 Chrysler hybrid minivans to be deploy over the self-driving car service Waymo (FCAW); moreover, the Waymo partnership is not the sole one: in 2016, an agreement between FCA and Google establishes a cooperation to develop a new generation of autonomous vehicles
\cite{fca_google} that will be ready for the market by the end of the decade. These collaborations are clearly representatives of the company's serious intentions to become a leader in this rather new driving paradigm \cite{fca_waymo_partnership}. With these cooperation, FCA aims to develop new forms of technologies that will enhance self-driving management tools, but also to acquire knowledge of possible new unexpected applications where autonomous vehicles could be suddenly employed.
It's worth noting that the self-driving phenomena has not only technological concerns, but also social ones: since nowadays cars are parked 95\% of the time \cite{fca_cars_95_parked}, self-driving cars could be a better option compared to conventional cars, meaning that a fewer number of cars can fulfill the requirements of an higher amount of customers. It has to be considered the reduced stress factor too, since driver would not be exposed to the parking stress attitude, nowadays so commonly spread out in crowded commercial zones. The results can be easily drawn: even if less cars will be built and sold, companies can raise the price of the single unit in order to guarantee more quality and way more tested features. Fewer cars and more quality would means less pollution and much more satisfied customers, so, a win-win for both the producer and the consumer categories. 
Although this situation could be considered a radical change, Waymo is one of the few that is prone to accept the risks, due to the experimental nature of habits conversion itself; FCA is aware of such controversial aspects too, but, nonetheless, is still ready to welcome every possible outcome as Waymo is, since the collaboration keeps going on strongly, having it been recently renewed\cite{fca_waymo_renew}.
