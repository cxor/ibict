\section{Tesla Motors}

\subsection{Business Model}
Tesla's founder and CEO Elon Musk launched the company with the mission of "accelerating the advent of sustainable transport, bringing electric cars en masse into the market as soon as possible." This sentence perfectly describes the big business model Tesla success.
Back in 2008, with the delivery of the first full electrical car named Roadster, Tesla entered the Electric Vehicles market, which probably was one of the less explored at that time.  
Nowadays, this company is able to exploit this market mainly because of its philosophy. In fact, Tesla Motors started developing its own cutting edge technology from the single battery cell to the engine gearbox, letting the company to rely on its own intellectual properties to build cars that shows incredible performances, not only in term of speed, but also in terms of battery degradation and range covered on a charge, showing to customers that this technology could be the future. 
It is important to note that, since all the Tesla’s intellectual properties are open source, every car manufacturer can further use and improve this technology.
Part of the Tesla’s business plan consists of the so-called “Masterplan” \cite{TeslaMasterplan}, used by the CEO Elon Musk to define the company’s strategy. As a first step, the company entered the market with high-end expensive cars to attract a relatively restricted circle of well-off customers (for example the Roadster was sold at 120'000\$ from the 2008 until the 2012). The second step consisted to release more affordable models, always respecting the high quality standard in terms of innovation; one prominent example is the Model S, available from 2012 and priced 65'000\$. In order to reach more customers, in 2017 the Model 3 was released starting at 35'000\$ and this represents the final step of the "Masterplan": being the price much affordable than preceeding models, Model 3 can clearly drastically broaden the market.
Along this selling strategy, the Business Model is composed by two other focal points.
In fact, the company provides a large and well developed network of charging point, the so-called “supercharging network”, which is very appreciated by the customers mainly because the company gives to them some free charges every year but also because it is ten time faster, i.e. about an hour, than charging the car at home.
Furthermore, one of the fundamental component of Tesla's business consists of the servicing facilities. Indeed, the presence of various centers all around the world permits the customers to request and buy repairs; moreover, newer models can be inspected remotely and specialized technicians can do maintenance directly at customers' house.

\subsection{Growth \& Vision}
Since Tesla's birth, its first goal was to incentivize the sustainable transport through a constant work aimed to create and refine the technology required to electric cars. Nonetheless, they are not only developing automobiles, indeed, in 2017 the company teased a new vehicle called Tesla Semi, which is an autonomous full electric truck with, as not fuel-powered vehicle, good recharge time and autonomy performances (around 800 kilometers). Furthermore, it boasts an energy consumption of less than 2 KWH per mile, which attracts the attention of many customers, interested in transport cost reduction. In fact, Semi, whose production should start in 2019, received pre-orders from various delivery companies.  
Moreover, the business of the company opened also to the house field with the presentation of the Tesla Powerwall \cite{Powerwall}, which is a big modular battery that can be installed at home and allows the storage of electrical energy produced by solar panels. Furthermore, on October 26th 2016, the CEO Musk announced a new product called “Tesla Roof Tiles” which consists of a root tile that can transfer the solar energy, through the means of electric energy, into an integrated Powerwall battery, providing it at any time.
The continuous growth of Tesla can be proved starting from its foundation in 2008, when the company was worth 14,7 million dollars. In fact, its value grew up to 111,9 million the following year with a 661\% surge. Successively, in 2010, Tesla went public with an initial public offering (IPO) of about 13,3 millions of shares at 17\$ each. Since then, the average value went up reaching about 360\$ per share at the beginning of December 2018, with a growth of around 2017\% in almost 8 years \cite{Growth}. 
In conclusion, the continuous development of electric batteries permits Tesla to broaden its views and try to become a good competitor in different markets, including, for example, the sustainable energy supply.
