\section{Tesla Motors}

\subsection{Business Model}
Tesla's founder and CEO Elon Musk launched the company with the mission of "accelerating the advent of sustainable transport, bringing electric cars en masse into the market as soon as possible." This sentence perfectly describes the big business model Tesla success.
Back in 2008, with the delivery of the first full electrical car named Roadster, Tesla entered the Electric Vehicles market, which probably was one of the less explored at that time.  
Nowadays, this company is able to exploit this market mainly because of its philosophy. In fact, the Tesla Motors started developing its own cutting edge technology from the single battery cell to the engine gearbox, letting the company to rely on its own Intellectual Property (IP) to build cars that shows incredible performances, not only in term of speed, but also in terms of battery degradation and range covered on a charge, showing to customers that this technology could be the future. 
It is important to note that, since all the Tesla’s IP is open source, every car manufacturer can use this technology.
Part of the Tesla’s business plan consists of the so-called “Masterplan” \cite{TeslaMasterplan}, used by the CEO Elon Musk to define the company’s strategy. As first step, the company entered the market with high-end expensive cars to attract the most affluent class of the society, achieving this with the Roadster sold at 120 000\$ from the 2008 until the 2012. The second step consists of releasing a more affordable car, always respecting the high standard of Tesla in terms of innovation, as the Model S, available from 2012 and priced 65 000\$. In order to reach more customers, in 2017 the Model 3 was released starting at 35 000\$ and this represents the final step of the "Masterplan".
Along this selling strategy, the Business Model is composed by two other points.
In fact, the company provides a large and well developed network of charging point, the so-called “supercharging network”, which is very appreciated by the customers mainly because the company gives to them some free charges every year but also because it is ten time faster, i.e. about an hour, than charging the car at home.
Furthermore, one of the fundamental component of Tesla's business consists of the servicing facilities. Indeed, the presence of various centers all around the world permits the customers to request and buy repairs; moreover, newer models can be inspected remotely and specialized technicians can do maintenance directly at customers' house.

\subsection{Growth & Vision}
Since Tesla's birth, its first goal was to incentivize the sustainable transport through a constant work aimed to create and refine the technology required to electric cars. Nonetheless, they are not only developing automobiles, indeed, in 2017 the company teased a new vehicle called Tesla Semi, which is an autonomous full electric truck with, as not fuel-powered  vehicle, good recharge time and autonomy performances. %It has an autonomy of 805 km and could be charged up to the 80\% in only 30 minutes. 
Moreover, the business of the company opened also to the house field with the presentation of the Tesla Powerwall \cite{Powerwall}, which is a big modular battery that can be installed at home and allows the storage of electrical energy produced by solar panels. Furthermore, on October 26th 2016, the CEO Musk announced a new product called “Tesla Roof Tiles” which consists of a root tile that can transfer the solar energy, through the means of electric energy, into an integrated Powerwall battery, providing it at any time.

%Tesla entered the market with the sport roadster. It was the first high-performance electric luxury sports car. The company sold around 2,500 Roadsters before finishing production in January 2012.
%Subsequently introduced the sedan, called the model S, in June 2012. Tesla then produced its first SUV, the X model, in September 2015. Tesla is also working on a new supercharged version of the Roadster, which they believe is the "the world's fastest car" which will be able to go 0-60 in 1.9 seconds. The release of this new Roadster is scheduled for 2020.
%Tesla started selling cars in the third quarter of 2012 delivering 321 cars in 3 months. Given the ability of the company to exploit the EV market and the trend of the society to move to more sustainable technology, these cars has become more and more popular at the point in which the deliveries hit the record of 83 500 cars in the third quarter of 2018.
%Back in the 2008, when in was founded, the company was worth “only” 14,7 million dollars; in the next year it growth to 111.9 million with a 661\% surge. Then in 2010 Tesla went public with an initial public offering (IPO) of about 13.3 millions of shares at 17\$ each. On the first day the price went up by the 40.53\% closing at near 24\$ per share. Since then, the average value went up until today (5th of December 2018) which is about 360\$ per share, meaning that the company value grew of the 2017\% in almost 8 years \cite{Growth}.

%A great moment of Tesla was the introduction of the full electric Semi-Truck in 2017. This self driven truck boasts an energy consumption of less than 2 KWH per mile. Semi received pre-orders from various delivery companies, with orders of more than 100 pieces. The production should start in 2019. Considering that about 30\% of companies that deal with transport do not run routes that exceed 320 km per day, knowing that there is a truck that can save fuel is certainly an opportunity that affects many companies.

%Despite the red accounts, the Tesla automaker is still growing and aiming higher and higher. Recently it has put on the market two versions of the Model 3: one with an autonomy of about 350 kilometers, capable to drive for 200 km charging only half an hour; the other with a battery capable of covering 500 km with a full charging cycle. Between the two there is also a difference in performance: the first hits 0-100 in 5.6 s against the 5.1 s of the second, and the top speed of 210 km/h top speed against 225 km/h. For 2019 Tesla has already envisaged the production of 500,000 units by acting significantly on the electric car market.

\bibliographystyle{plain}
\bibliography{references/tesla_BM_G_V.bib}