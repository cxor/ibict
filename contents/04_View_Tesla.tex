\section{View 2}
\subsection{Business Model \& Vision}

Tesla's founder and CEO Elon Musk launched the company with the mission of "accelerating the advent of sustainable transport, bringing electric cars en masse into the market as soon as possible." This sentence perfectly describes the big business model Tesla success.
In 2017, almost 1.2 million electric cars were sold globally, with a 57\% increase compared to 2016 (around 750 thousand) and more than double the 537 thousand electric cars of 2015, but the most recent estimates they claim that in 2040 at least 800 million ecological cars will be registered. Tesla is among the pioneers, thanks to the unconventional move to make patents open source. The trend of electric cars is certainly positive and, according to forecasts, it should continue also in 2019, with almost two million new electric vehicles expected on the market.

Tesla took a unique approach to putting its first vehicle in the market. Instead of trying to build a relatively cheap car that could be mass produced in the market, he took the opposite approach, focusing instead on creating a convincing machine. In a post on Tesla's website, CEO Elon Musk described the company's mission in this way: "If we could have our first product, we would have, but that was simply impossible to achieve for a startup company that had never built a car and that had one technology iteration and no economies of scale. It was like this, but we decided to build a competitive position with its gasoline alternatives ".



\subsection{Growth}

Tesla entered the market with the sports roadster. It was the first high-performance electric luxury sports car. The company sold around 2,500 Roadsters before finishing production in January 2012.
Subsequently introduced the sedan, called the model S, in June 2012. Tesla then produced its first SUV, the X model, in September 2015. Tesla is also working on a new supercharged version of the Roadster, which they believe is the "the world's fastest car" which will be able to go 0-60 in 1.9 seconds. The release of this new Roadster is scheduled for 2020.

A great moment of Tesla was the introduction of the all-electric Semi-Truck in 2017. This autopilot truck boasts an energy consumption of less than 2KWH per mile. Semi received pre-orders from various delivery companies, with groups that ordered 125 copies. The production of the seeds should start in 2019.

Despite the red accounts, the Tesla automaker is still growing and aiming higher and higher. Recently it has put on the market the Model 3, of two types: one with an autonomy of about 350 kilometers, capable, in half an hour of recharge, to restore about 200; the other with a battery capable of covering 500 km with a cycle. Between the two there is also a difference in performance: 0-100 in 5.6 '' against 5.1 '', 210 km / h top speed against 225. Considering that about 30\% of companies that deal with transport do not run routes that exceed 320 km per day, knowing that there is a truck that can save fuel is certainly an opportunity that affects many companies. Tesla has already envisaged the production of 500,000 units by acting significantly on the electric car market.