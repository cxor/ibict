\section{Introduction}
% This section should give an introductory explanation regarding what is the problem faced in your battle. Is the problem more technological, socio-economical, or both?\\ Also introduce which are the two different points of view of the scenario, the two factions, and which problems and solutions they want to focus on.

Innovation can be undeniably seen as engine which allows the world to progress and enhance our living conditions, but when it comes to business, having innovative ideas isn't always enough to reach and maintain success.

Douglas Hofstadter once said that we are "strange loops"; behind this assertion there are many interpretations, but surely enough one lies under the fact that we are continuously trying to ask ourselves the same questions again and again, sometimes concluding to not have a complete, precise idea of what the answers could be.

As many entrepreneurs did during the centuries, we asked ourselves which are the keys to manage and integrate innovation in the business world. In order to better understand the nature of such "ingredients", we tried to highlight some questions which could led us to a hopeful enlightenment. At a first glance, there were quite a few questions that pounded our minds, for instance: are we always prepared to confront us, giving the rapid advancement in technology, with rapid integration of innovation in our lifestyle? Or we should prefer a more slow-paced, reflective and time-tested approach? Is innovation -at least sometimes- creating more problems than solutions?
How we should react to unpredictable (and unwanted) effects that some kind of innovation can lead to, giving that the innovation itself brings several benefits?

Of course trying to answer to all this questions is a fairly hard investigation, and even after a statistically sound analysis we could not have clear ideas about their answers. By the way, as a matter of fact, we will use these interrogatives as guidelines to have a better grasp of why innovation itself isn't merely sufficient to conduct a winning business.

In this work we present a case study where we compare two distinct business models; in particular, we restrict our focus on the automotive world, considering two eminent companies: Fiat Chrysler Automobiles (from now on simply "FCA") and Tesla Automobiles (hence forth called "Tesla"). The first one is representative of a more conservative car manufacturer, which integrates innovations and technologies in a slow and more financially safe manner; on the other side, Tesla brings the flag among those companies which are on the bleeding edge when it comes down to innovation and its rapid and proud integration in nowadays vehicles.
