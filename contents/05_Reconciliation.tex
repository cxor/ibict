\section{Reconciliation}
% Here is where your report should focus on: the reconciliation between the two views. As you know, when dealing with ambiguity it’s impossible to agree with one opinion or another. You probably saw this when we conducted polls in the class: sometimes we had strong differences, sometimes they were way more balanced. In ambiguity you base your decision on your preferences, your beliefs, your values, and then you use the retrospective rationality to justify them… However, is there only one “right” side? How can you assess what is right and what is not? Often, the result is in the middle, in a reconciliation between the two extreme views, where there is no absolute winner or loser. This is the work which we ask of you here. In this reconciliation you should create a third story, what we call a “fair tale”, that merges the two “fairy tales” which you told when the views were opposed. Try to maximize the upsides of each view and minimize the downsides… What emerges? In a Hegelian fashion, this is a synthesis that has been enriched by the antithesis, and which is more than a mere compromise between the two parts.


self-driving vs normal
    All-in investment vs restricted investment, partnership with Waymo.

safety and security
maintenance
environmental sustainability


vision
    FCA bets on short-term innovation. This means hybrid vehicles, or gas ones. Tesla, on the other hand, is heavily investing in fully electrical vehicles, presenting a bleeding-edge technology plateau and a short square blanket methodology of investment.
    
business model
    FCA have shown that the Global Manufacturing System has a massive beneficial impact on the efficiency of production management. The research is not quite as strong as that carried out by Tesla engineers, but this activates the "reuse when ready-to-market" strategy to be steadily adopted.
    
electric vs normal
    It's clear that electric car are way more future-sustainable than their classic counterpart. That said, it's also quite obvious that the former category brings many issues that cannot be ignored, most of all those concerning the environment and maintenance. When it comes down to self-driving technology, there is a staggering amount of research that is undergone by many big players, and Tesla is surely belonging to such group; the stage of development is quite mature, but at the same time it can be considered solidly stable since the risk factors are excruciatingly heavyweight, being directly bound to people safety and security.
    
growth
    {ref to bath tubes curves, usw} FCA shows a growth which is basically linear in time, slow but constant. It's mature and well established market guarantees a consistent margin of profit, but requires also less investments in research and technology integration and testing. Tesla, being in an early stage of development, has to face a bigger amount of investments for the maintanance of its infrastructures, but also has a significantly higher expenses due to research purposes. It's clear that the growth potential here is much more explosive in comparison to FCA's, but it's of the same importance to understand how much the issues derived from innovation will lead to slow down or even break such growth. (There can be many other companies that will benefit from Tesla researches, and then enter the market later -on purpose, obviously- in order to cut out all the expenditure necessary to establish a new ignition paradigm. 




