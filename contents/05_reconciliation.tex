
\section{Reconciliation}
% Here is where your report should focus on: the reconciliation between the two views. As you know, when dealing with ambiguity it’s impossible to agree with one opinion or another. You probably saw this when we conducted polls in the class: sometimes we had strong differences, sometimes they were way more balanced. In ambiguity you base your decision on your preferences, your beliefs, your values, and then you use the retrospective rationality to justify them… However, is there only one “right” side? How can you assess what is right and what is not? Often, the result is in the middle, in a reconciliation between the two extreme views, where there is no absolute winner or loser. This is the work which we ask of you here. In this reconciliation you should create a third story, what we call a “fair tale”, that merges the two “fairy tales” which you told when the views were opposed. Try to maximize the upsides of each view and minimize the downsides… What emerges? In a Hegelian fashion, this is a synthesis that has been enriched by the antithesis, and which is more than a mere compromise between the two parts.


The battle allowed us to face a direct comparison among two fundamentally different companies. It is clear that one of them, namely FCA, has a more "conservative" approach on how to manage business; on the other hand, Tesla, being at an early stage of evolution, has a way more aggressive attitude towards the business directives. Let's analyze in detail how we can reconciliate the just depicted realities.
Tesla vision is arguably more socially approvable, since the main idea is to definitely solve the problem sustainable transportation; FCA vision is also clear, but in our opinion is more investor-friendly: the strategic choice to employ great amount of resources in short-term solutions can lead to faster and more prominent market approval, thus resulting in higher profits. 
Having future-proof research directives is also crucial, but even more so is how such research is carried out: FCA approach makes use of many cooperations with a distinct range of partners, Tesla prefers to develop its own solution in-house; from our point of view, since both of them have their downsides, the best approach would be to maintain a right amount of high-potential projects private, especially the ones that are seen as more promising, and a fairly good quantity of collaborations with external partnerships, thus enabling the possibility to gain the market share which is already targeted by the partner company itself.
Taking the self-driving technology as an example, is evident that FCA is not even close to develop something comparable to Tesla's, despite the double partnership with Waymo and Google; such efforts, when compared to what Tesla is investing in order to create next generation systems, could appear limited, or even inappropriate. We think that it is not the case. The rationale behind that can be found in \emph{what} the respective companies are currently investing on: FCA, during the last 10 years has undergone some dramatic changes in its internal structure, strictly focusing on making the whole production process more efficient and, at the same time, less expensive. The World Class Manufacturing is employed to make loss more mathematically predictable, thus the margin profits can be estimated in a more accurate fashion when compare to a company that has considerable expenses due to the engineering aspects of innovations. We think that a company like FCA could fit a brand new technology in its own production arsenal in less time and with minor effort when compared to something like Tesla, since the latter does not have at its disposal the necessary structure to do the same. 
Balancing how much is required to invest in technology research and how, on the other side, is going to be used for business integration its a focal point to conduct a successful business. Because Tesla produces most of its technology directly in house, business integration is not a main concern as today, but could become so in the upcoming years. The intention to bring the recycling process within the Giga-factory is one distinct factor on how much financial pressure the company is now facing, probably because the expenditure for processes like research and testing is so ingent that there is no way other than trying to minimize the usage of external entities. Given the above reflections, an investor could easily judge the potential growth of a company based on the FCA model as "promisingly constant, slightly varying in the amount of profit it will bring over time"; on the contrary, Tesla's paradigm of doing business can be recognized as riskier, but also capable of much higher revenues given the major technical potential.
Our final judgment sees the FCA business model still more adequate for an investment because, although it looks less promising, can be better integrated in an existing business, therefore potentially providing more occasions to create a diverse net of new investments. 
What about innovation to market times? We reckon that fully-electrical vehicles are the imminent future; this transient phase sees LPG-fueled vehicles in a kind of golden age, but predictions also show that their adoption will not massively increase, especially because the volatility (and unpredictability) of gas prices and the continuous improvement related to engines electrification. FCA is slowing transitioning towards electrical ignition but, as investors, we have would like to see much more solutions reaching the market by the end of the following year, and not a bunch of them just to test the ground.
For what concerns safety and security, it is awkward to compare FCA to Tesla: the latter vehicles are in a much higher price range than the average FCA most sold vehicle; it is fairly understandable that a Model S has tons of sensors and features that help to create a pleasant, relaxed guide experience than, let's say, a Fiat Panda. We think that both the companies are doing well in this respect, and it is not an insult to consider this a draw, in terms of gameplay. Maintenance is completely different story: in this case the situation expose a company -Tesla- whose aim is to reduce as much as possible the necessary maintenance services against one -FCA- which strives to achieve outstanding services, betting hard on a centralized organism that takes care of all its concerns. An investor should not only understand what a customer can love about a product, but also all the surrounding aspects that make it unique: in the automotive realm, for instance, maintenance services are game changer, because are the main point that will lead a customer to become an aficionado, satisfying its need when problem arise above all. Under this perspective, FCA is doing a great job; Tesla, also consider -as already stated- its early "career", is also pushing notable efforts (for example the technician-at-home solution: in one word, brilliant).