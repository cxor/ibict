\subsection{Safety}

Safety and security have always been focal points of research within Tesla. Just to name few self-speaking results, every Tesla model released on the market has reached at least a 5 star rating out of 5 in the EuroNCap crash test, with the Model S, Model X and Model 3 outclassing the standard rating scale achieving an outstanding 5.4 out of 5 rating \cite{tesla_euroncap}.
By NHTSA standards (National Highway Traffic Safety Administration, the american agency monitoring safety measures in automotive trasportations) all Tesla cars set a new record in terms of safety -having the lowest probability of injury-, outclassing even recent Volvo cars, which historically have been leaders in the sector. For example, when compared to the Volvo S60, Tesla Model S preserved 63.5\% of driver residual space against the 7.8\% of the Volvo counterpart which, among other things, uses one of the most innovative solutions available (similar in characteristic to the one employed in the Apollo Lunar Lander) \cite{}.
Furthermore, electric vehicles are safer and more reliable by construction, since there is a drastic reduction in the number of moving and vibrating parts. Thus, their maintenance is greatly simplified, and also considerably cheaper \cite{}. Many concerns have been arose around the battery, since it's considered the most critical part in the entire vehicle structure: Tesla, in order to guarantee the highest level of safety possible, implements state-of-art technologies against overheating or burning danger. To better clarify how much the company cares in matter of safety and security, it's known that it has instructed firefighters to deal with a burning Tesla car \cite{}. Some innovations introduced in recent models allow to shift the car in autopilot mode whenever the board computer identifies some kind of danger, whether -for instance- the driver falls asleep or faints.
It's worth spending a brief commentary on the weaknesses of Tesla vehicles regarding advanced digital assets at their disposal. Cybersecurity, within the automotive world, is a pretty novel topic, and, although rather intuitively crucial, is not always taken seriously by vendors. On the contrary, Tesla was one of the first among all that has decided to adhere to severe guidelines of conduct in order to reduce the number of exploitable vulnerabilities; has also funded numerous campaigns to reward anyone who finds a new weakness in the digital systems adopted \cite{tesla_security_methods}.

\section{Self-Driving}

Tesla is arguably the absolute leader in self-driving technology. Its "Autopilot" toolset, now at the 9th version, is the most trusted self-driving system around \cite{tesla_autopilot_trust}. Standing on official declarations, this system allows to reach a full, driverless, control of the vehicle, but, because of the limited time spent in testing, it is classified only as a "drive assistant", thus still relying on the presence of a physical driver \cite{tesla_full_selfdriving}.
The nature of the driverless paradigm is quite a complex issue. As reported in \cite{tesla_autopilot_awareness}, as of 2018 consumers are excited by the self-driving phenomenon, but also distrustful in its sudden adoption. Moreover, it shows that people generally prefer control over automation, and would be quite reluctant to get in an autonomous car \cite{tesla_pilot_cheatsheet}. Although this situation seems to pose Tesla in a disadvantageous position, it is also true that is allowing the company to have the required time to deepen and extend the safety tests, thus gaining an increasing level of trust from the market, as well as social consensus.
