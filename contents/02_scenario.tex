\section{Scenario}
% Explain the “storyline” of the scenario and the “battlefield rules” that were negotiated. Reference explicitly what you agreed would be inside and outside the boundaries of the battle and the decision process that went behind it. In a Hegelian triad, this is a shared thesis.

As for all the "battles" of the IBICT course, the comparison was carried out by two teams, one taking the side of FCA, the opponent impersonating Tesla Automotors.
The playground for the battle was prepared assuming that an audience of investors were willing to invest in one of the two companies selected, where each team had the role to convince the audience itself about the presumed superiority of a company in respect to the other.
Investors are generally interested in risk levels, past performances, future projections and obviously profitable margins: as a result, the comparison have been imprinted on focal points directly related to such aspects. At a first glance was necessary to explain the technology stack that each company is pushing as its working flag: that means looking side by side two considerably different parties, where the first's concerns is to improve the existing and well-tested driving models (FCA), the other focusing on bleeding-edge innovations, with fully-electrical ignition and self-driving algorithms on the forefront of the production line. Once understood the technical background of the competitors, it is also meaningful to discuss about the business model built upon the marketing of such technologies, the vision behind their wide-spreading and the growth that they promise, as well as some collateral topics, such as maintenance services, environmental sustainability and driving safety and security.
Of course there are many other significant comparison points to talk about; nonetheless, we considered at unanimity the ones above to be essential in order to allow an investor to choose which company deserves to be financially supported. Just to cite a prominent example between the excluded topics, leadership and CEO has been decided to be cut out from the discussion since Elon Musk has been recently accused of fraudolent activity; giving that both FCA and Tesla are far away from being one-man companies, is logically understandable that, in order to maintain a certain level of fairness during the debate, it's mandatory to avoid subjects that can potentially be "slippery" for one team and a kill switch for the other.


%The first battle that we have experienced, was underwhelming. Although the teams were prepared, only few attacks were made and the questions from the public saved the lesson. To not suffer the same fate we sat down and devised specific rules to guarantee  a fierce and enjoyable battle.
%First, the audience identity was decided. We chose to address the students as \textbf{\textit{potential investors}}, since we wanted to get rid of brand bias which was going to happen otherwise. The next logical step was to identify what makes a good investment.
%To our best knowledge, investors are most interested in the level of risk, past performance, future projections and obviously business plan of a company.
%For starters it was decided to analyze in depth the
%\textbf{\textit{business plan}}, the projected \textbf{\textit{growth}} and \textbf{\textit{vision}} of both companies for the near and far future. 
%The discussion on different \textbf{\textit{business plans}} naturally brought up the topic of \textbf{\textit{electric and ICE cars}}.
%Also, since the future of ICE cars is uncertain, it was decided to include also the discussion on  \textbf{\textit{environmental sustainability}} of each company.
%We have also analyzed the trends and hot topics in the automobile industry and came to the conclusion that it is a must to discuss \textbf{\textit{self-driving}} strategies, their \textbf{\textit{safety}} as well as the overall \textbf{\textit{safety}} of the vehicles.
%As our society is always more connected it was decided to research also product \textbf{\textit{security}} that both companies provide.
%\newline For fair-play and due to our engineering background it was decided to focus mostly on a more technological approach to the modus operandi of both companies, to exclude to the major extent the \textbf{\textit{CEO's}} and to consider only \textbf{\textit{information older than 2 weeks}}.
%\newline From the get-go, Tesla team was facing an uphill battle since the company financial struggles, instability and production problems are common knowledge.
%\newline On the other hand FCA had to prove that even though they were late to enter the EV market and are years behind on safety and innovation, are still competitive and with a plan for years to come even without the leadership of late Sergio Marchionne.

