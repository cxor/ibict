\section{Scenario}
% Explain the “storyline” of the scenario and the “battlefield rules” that were negotiated. Reference explicitly what you agreed would be inside and outside the boundaries of the battle and the decision process that went behind it. In a Hegelian triad, this is a shared thesis.

Setting a scenario in order to compare two big companies such as FCA and Tesla is not an easy task; there are plenty of options that we could choose, and that can possibly dramatically vary our judgement on how suitable is a certain way to conduct business. How can we assign a meaning to our comparison, and how we can be sure to settle some boundaries to not loose focus?
In order to tackle this situation, it was established that the battle will have seen from the eyes of potential investors who are willing to financially support either one or the other company on the basis of facts and rethorics. \n
Investors are generally interested in investments' risk levels, past performances, future projections and obviously profitable margins. To fairly face aspects that are big concerns of both, we initially thought about which are main tecnological challenges that nowadays car producers are working on, then we move on to collateral issues which of course are not less important, but just a derivation from the previous category. The former includes novel paradigms of ignition, seeing a confrontation between the traditionally gas-driven vehicles with the more or less new fully electrical paradigm. Once undrstood that these inner working mechanisms have very few common points, it's straightforward to ask ourselves what are the differences around them; this translates into how and with wath reliability maintance is provided by vendors, whether the production of vehicles and their emissions/consumptions offer environmentally sustainable approaches.
Moreover, other technologies advacements are provided by the self-driving innovation; since it's clear that this is an hot topic in the automotive world, we reckoned at unanimity how important is to further discuss the implications of an its early adoption.
Given all of these concerns, two technological views bring two radically different business models. How companies are managing their business is directly tied to the technology they are selling, so it's interesting to compare their respective visions and how can these lead to the actual growth. \n
To recap: at a first glance we will look at the technologies that each company are pushing as their working flags; then, we will discuss about the business model build upon the marketing of such technologies, the vision behind their widespreading and the growth that they promise. Finally, we will go through how the respective sides guarantee a proper maintanance service and what is their environmental impact and measuring its sustainability threshold. /// mettere questa parte come intro; dopodiche' utilizzare la parte precente come spiegazione dettagliata del perche' abbiamo scelto questi punti.

The first battle that we have experienced, was underwhelming. Although the teams were prepared, only few attacks were made and the questions from the public saved the lesson. To not suffer the same fate we sat down and devised specific rules to guarantee  a fierce and enjoyable battle.
First, the audience identity was decided. We chose to address the students as \textbf{\textit{potential investors}}, since we wanted to get rid of brand bias which was going to happen otherwise. The next logical step was to identify what makes a good investment.
To our best knowledge, investors are most interested in the level of risk, past performance, future projections and obviously business plan of a company.
For starters it was decided to analyze in depth the
\textbf{\textit{business plan}}, the projected \textbf{\textit{growth}} and \textbf{\textit{vision}} of both companies for the near and far future. 
The discussion on different \textbf{\textit{business plans}} naturally brought up the topic of \textbf{\textit{electric and ICE cars}}.
Also, since the future of ICE cars is uncertain, it was decided to include also the discussion on  \textbf{\textit{environmental sustainability}} of each company.
We have also analyzed the trends and hot topics in the automobile industry and came to the conclusion that it is a must to discuss \textbf{\textit{self-driving}} strategies, their \textbf{\textit{safety}} as well as the overall \textbf{\textit{safety}} of the vehicles.
As our society is always more connected it was decided to research also product \textbf{\textit{security}} that both companies provide.
\newline For fair-play and due to our engineering background it was decided to focus mostly on a more technological approach to the modus operandi of both companies, to exclude to the major extent the \textbf{\textit{CEO's}} and to consider only \textbf{\textit{information older than 2 weeks}}.
\newline From the get-go, Tesla team was facing an uphill battle since the company financial struggles, instability and production problems are common knowledge.
\newline On the other hand FCA had to prove that even though they were late to enter the EV market and are years behind on safety and innovation, are still competitive and with a plan for years to come even without the leadership of late Sergio Marchionne.

