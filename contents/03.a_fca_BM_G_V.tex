\section{Fiat Chrysler Automobiles}


\subsection{Business Model}
The two holding companies for automobiles production, Fiat and Chrysler Groups, became a multi-national organization when they merged in early 2014 \cite{fca_history}. With business coverage spread over 140 countries, an employment of nearly 236,000 people, this strategic fusion lead FCA (i.e. "Fiat Chrysler Automobiles") to be one of the biggest automotive manufacturers in the whole planet. The FCA group now has control over the whole chain of production and services, including design, engineering, manufacturing, vehicles and spare components sales, and provision of a multitude of services \cite{fca_overview}. The business operativity is active worldwide through the 87 research and development centres, 159 manufacturing facilities, as well as a multitude of dealers and distributors. The production of technology has been driven by the sub-brand Magneti Marelli, which has provided FCA with the opportunity of having consistent and innovative development of vehicle components and low dependency from third party supplies \cite{magneti_marelli}. Other businesses within the FCA group further contribute to the complete control of the vehicle production, including Comau for the development of production systems and sub-brand for iron and castings, Teksid \cite{fca_overview}.\n 

Mopar is also an integral part of the FCA group, previously being a part of Chrysler Corporation. Mopar, whose name stands for "MOtor PARts",finds its target on  offering quality accessories and aftermarket parts for FCA vehicles as well as maintenance services for owners \cite{fca_mopar}. Being FCA focused on production efficiency, the choice to have a dedicated sub-brand in order to provide this kind of services is not unexpected. The first and most obvious advantages are the working focus and the tasks management that, being it handled in a centralized way, is crucially improved in how efficiently it is carried out. Furthermore, the guaranteed compatibility of replacing components is something that no other aftermarket company can provide \cite{fca_mopar}. Finally, the services provided to owners, such as vehicle protection plans, roadside assistance and express care should ensure that the vehicle is safe and in perfect order. These efforts are signs that FCA is striving to create standardization, quality and services that are close to the customer and make use of FCA's high presence in worlwide markets.

The latter aspect probably is, in fact, one of the most prominent FCA marketing strategies: currently, the company delivers vehicles to customers through the brands Abarth, Alfa Romeo, Chrysler, Dodge, Fiat, Fiat Professional, Jeep, Lancia, Ram and Maserati \cite{fca_overview}. Within this portfolio of brands, FCA can offer a wide variety of vehicles in a number of different segments. This includes passenger cars ranging from the A-segment (i.e. city cars), where FCA has a history of successes, such as the Fiat Panda and the Fiat 500, all the way up to luxury F-segment cars, such as the Maserati Quattroporte, but also sport utility vagons where Jeep is a well known player, multi-purpose vehicles with the Dodge Caravan as an example, the pickup-truck dedicated brand Ram, labour vehicles such as Fiat Fiorino and Alfa Romeo or Maserati when it comes down to racing vocations.
The choice to maintain a high level of diversity in the production regime ensures a better penetration in different demographic and geographic zones; this strategy also allows to keep revenues at a constant level throughout the year, making marketing predictions much more accurate and realistic \cite{fca_marketing_strategy}.


\subsection{Strategic growth and vision}
Is not a secret that the two main groups that compose FCA have had hard times during the decade 2000-2010.
Since then, FCA operational directives were strategically focused upon a completely new, fully (economically) sustainable approach. Most of this changes were applied by the former company CEO Sergio Marchionne \cite{fca_marchionne_fortune}, who has restrengthened the Fiat brand by merging with Chrysler, optimizing resources and processes, exiting low-revenue markets and rearranging the personnel structure, improving teaming and cooperation \cite{fca_marketing_strategy}. Financially speaking, the results of such complete overhaul have been impressive: FCA was transformed from a almost zero growth company into one of the biggest giants in the automotive industry in less than ten years, leading to a staggering increase in company revenues: in 2017 alone, net revenues almost doubled the preceding year. \cite{fca_marchionne_fortune}\cite{statista_fca_net_revenues}.

The report of the business plan from June 2018 gives an indication of the strategy for the company the following four years up until 2022.The report itself is a clear indicator of how the company is targeting production efficiency in order to not only improve revenues, but also to tackle the environmental and technological aspects \cite{fca_financial_overview}. The key that allowed to massively raise profits and, at the same time, to advance in both these fields, is called "World Class Manufacturing" (WCM). The WCM is a framework of methodologies inspired by the Japanese lean production philosophy \cite{fca_wcm}, has allowed  to double the revenues in the past years (as already stated above) and will permit to further increment them in the upcoming ones. 
The reasons why the WCM has been and will be so successful can be found in two simple words: sustanaibility and manageability. Within the optics of the WCM, a working team face every decision is going to take (maintanance, logistic, quality assurance, safety, management, ...) on the basis of their economical incidence. The team activity is oriented toward the accomplishment of small, focused projects (Kaizen) whose objectives are "zero defects, zero faults, zero incidents, zero remaining stocks": in other words, the final target, which is a consistent reduction of the production plant costs, is ensured by the autonomy of the team itself (that's why decisions are "manageable"), using parameters that are always taking into account the economical aspect (and thus "sustainable").

Speaking about near term technological investments, the current business plan introduces a forecast regarding the electrification of vehicles within the FCA group. In the last years, FCA has displayed skepticism regarding the electric cars \cite{fca_electric_bet}, but for the upcoming years, substantial investments in the sector will allow to introduce the proper technology with the aim to provide FCA a place between the electrical automotive leaders in the world \cite{fca_financial_overview}. Because electrical ignition has several open issues on its side, where battery production and disposal are questionably only two of the most dangerous ecological problem nowadays, FCA has preferred a more conservative approach towards such innovation. The current marketing strategy \cite{fca_financial_overview} seems to dictate a phase of analysis and careful observation, where the company itself is both undertaking plenty of research and at the same time managing to interweave strategical partnership (such that with Waymo in recent times \cite{fca_waymo_partnership} in order to gain control over innovations that have already gained an acceptable level of maturity.


\bibliographystyle{plain}
\bibliography{references/fca_BM_G_V.bib}