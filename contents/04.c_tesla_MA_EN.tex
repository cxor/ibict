\subsection{Environmental Sustainability}

Few people know that Tesla has been founded by pioneers such as Martin Eberhard and Marc Tarpenning because they wanted to avoid the disposal of electric cars produced by General Motors, back in 2003. \cite{muskGM} This simple "recycle" purpose was further transformed into a visionary target: \emph{"accelerate the world’s transition to sustainable energy through increasingly affordable electric vehicles and energy products"}, as showed by the official Tesla site \cite{aboutTesla}.

The project of environmental sustainability starts right from the factories, or Giga-factories as Tesla loves to call them. These facilities are designed to be fully powered using renewable energy; moreover, the production processes are highly optimized to harness the most modern energy conservation techniques, as well as less manual intervention as possible. In this way the overall result is the most effectively achievable with nowadays instruments. These two factories are located one near Sparks, Nevada and one in Buffalo, New York, and are used respectively to produce lithium-ion batteries and solar panels; another one will be built in Shangai, China. Tesla is also active in the production of renewable energy toolset, such as solar panels which the current production is demanded to the subsidiary SolarCity.

Considering that the vehicles produced by Tesla are only powered by electric energy, it is rather clear that the margin of emissions' reduction can be fairly huge; despite this fact, another substantial problem must be taken into account: battery production and disposal. Batteries are one of the most important element within the overall fully-electric architecture of Tesla cars, so it is not possible to simply find temporary workarounds that limit the problem, but do not solve it entirely. In fact, their construction requires high quantity of greenhouse gas -a non-renewable resource-, and more environmental pollution is generated by extraction of minerals needed to properly build energy-storing cells \cite{scheele2016cobalt}.

Tesla has currently many plans to tackle such issues, and some of them are under a testing period; today, the environmental impact of Tesla batteries is mitigated by recycling batteries through the means of partner companies, with prospects to do it within the Giga-factory alone by the end of a couple of years, as the CTO Straubel said \cite{recycleBattery}.

\subsection{Maintenance}

The motor of a Tesla car has less moving parts than the one of a fossil fuel powered one, thus greatly improving its duration and drastically reducing the possibility of a malfunction. Because of these unique characteristics, maintenance of Tesla vehicles is way easier and cheaper. A picture of a Tesla motor -which lasted for over one million miles in a test- was published on Twitter, and the exterior appearance is basically the same of brand new one\cite{tesla1mMiles}.

A Tesla car has a 4 years (or 50000 miles) warranty, which covers repairs or replacements needed to correct defects in the materials or any part manufactured or supplied by Tesla which come with normal use. Moreover, the battery and the drive unit of the car are covered by an extended 8-year warranty, also ensuring free of charge reparation from a battery fire -if it ever occurs!-, even in the case it's the result of a driver's error \cite{teslaWarranty}. 

Tesla also offers some optional maintenance plans, which cover the annual service inspections for 3 or 4 years. Although such assistance is not available worldwide, Tesla is pushing great efforts in order to broaden the covered areas. 

For any problem with their car, accidental or not, Tesla offers service centers placed all over the world, with their number in continuous expansion; it's also possible to book overhauls directly from the onboard computer. A more convenient alternative is also available: if required (especially in zones far away from the nearest center), an expert technician can be sent over to the customer's house and carry out the needed operations right on the place.