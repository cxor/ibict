\section{Conclusions}
% In this section you should give a clear idea about why the reconciliation is needed and justify your fair tale choices. Here you should also refer to the concepts exposed in class. Which are the topics we described during theoretical lessons that fit in your scenario? Also, reflect on the impact in the real world of the problems and solutions you discussed in your battle. Was the topic of the battle completely explored? Should we plan for a follow-up of some sorts in future courses?

Creating a business is as complex as keeping it going strong. Predictability and risk, technology creation and integration, strategy and cooperation: these contrasting concepts are the weights that balance a delicate mechanism. The analysis done in the reconciliation section allowed us to not consider traditional, conservative approaches to business not fashionable anymore, but it lead us to the conclusion is that there are so many variable in this beautiful equation that is nearly impossible to predict its result with too many unknowns. An innovative idea alone does not mean a success when reaching the market, as well as a refined production system that has provided glorious revenues in the past but not integrating new bleeding-edge features will certainly not last forever.