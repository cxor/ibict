\section{Fiat Chrysler Automobiles}


\subsection{Business Model}
The two holding companies for automobiles production, Fiat and Chrysler Groups, became a multi-national organization when they merged in early 2014 \cite{fca_history}. With business coverage spread over 140 countries, an employment of nearly 236,000 people, this strategic fusion lead FCA (i.e. "Fiat Chrysler Automobiles") to be one of the biggest automotive manufacturers in the whole planet. The FCA group now has control over the whole chain of production and services, including design, engineering, manufacturing, vehicles and spare components sales, and provision of a multitude of services \cite{fca_overview}. The business operativity is active worldwide through the 87 research and development centres, 159 manufacturing facilities, as well as a multitude of dealers and distributors. The production of technology has been driven by the sub-brand Magneti Marelli, which has provided FCA with the opportunity of having consistent and innovative development of vehicle components and low dependency from third party supplies \cite{magneti_marelli}. Other businesses within the FCA group further contribute to the complete control of the vehicle production, including Comau for the development of production systems and sub-brand for iron and castings, Teksid \cite{fca_overview}.\n 

Mopar is also an integral part of the FCA group, previously being a part of Chrysler Corporation. Mopar, whose name stands for "MOtor PARts",finds its target on  offering quality accessories and aftermarket parts for FCA vehicles as well as maintenance services for owners \cite{fca_mopar}. Being FCA focused on production efficiency, the choice to have a dedicated sub-brand in order to provide this kind of services is not unexpected. The first and most obvious advantages are the working focus and the tasks management that, being it handled in a centralized way, is crucially improved in how efficiently it is carried out. Furthermore, the guaranteed compatibility of replacing components is something that no other aftermarket company can provide \cite{fca_mopar}. Finally, the services provided to owners, such as vehicle protection plans, roadside assistance and express care should ensure that the vehicle is safe and in perfect order. These efforts are signs that FCA is striving to create standardization, quality and services that are close to the customer and make use of FCA's high presence in worlwide markets.

The latter aspect probably is, in fact, one of the most prominent FCA marketing strategies: currently, the company delivers vehicles to customers through the brands Abarth, Alfa Romeo, Chrysler, Dodge, Fiat, Fiat Professional, Jeep, Lancia, Ram and Maserati \cite{fca_overview}. Within this portfolio of brands, FCA can offer a wide variety of vehicles in a number of different segments. This includes passenger cars ranging from the A-segment (i.e. city cars), where FCA has a history of successes, such as the Fiat Panda and the Fiat 500, all the way up to luxury F-segment cars, such as the Maserati Quattroporte, but also sport utility vagons where Jeep is a well known player, multi-purpose vehicles with the Dodge Caravan as an example, the pickup-truck dedicated brand Ram, labour vehicles such as Fiat Fiorino and Alfa Romeo or Maserati when it comes down to racing vocations.
The choice to maintain a high level of diversity in the production regime ensures a better penetration in different demographic and geographic zones; this strategy also allows to keep revenues at a constant level throughout the year, making marketing predictions much more accurate and realistic \cite{fca_marketing_strategy}.


\subsection{Strategic growth and vision}
Is not a secret that the two main groups that compose FCA have had hard times during the decade 2000-2010.
Since then, FCA operational directives were strategically focused upon a completely new, fully (economically) sustainable approach. Most of this changes were applied by the former company CEO Sergio Marchionne \cite{fca_marchionne_fortune}, who has restrengthened the Fiat brand by merging with Chrysler, optimizing resources and processes, exiting low-revenue markets and rearranging the personnel structure, improving teaming and cooperation \cite{fca_marketing_strategy}. Financially speaking, the results of such complete overhaul have been impressive: FCA was transformed from a almost zero growth company into one of the biggest giants in the automotive industry in less than ten years, leading to a staggering increase in company revenues: in 2017 alone, net revenues almost doubled the preceding year. \cite{fca_marchionne_fortune}\cite{statista_fca_net_revenues}.

The report of the business plan from June 2018 gives an indication of the strategy for the company the following four years up until 2022.The report itself is a clear indicator of how the company is targeting production efficiency in order to not only improve revenues, but also to tackle the environmental and technological aspects \cite{fca_financial_overview}. The key that allowed to massively raise profits and, at the same time, to advance in both these fields, is called "World Class Manufacturing" (WCM). The WCM is a framework of methodologies inspired by the Japanese lean production philosophy \cite{fca_wcm}, has allowed  to double the revenues in the past years (as already stated above) and will permit to further increment them in the upcoming ones. 
The reasons why the WCM has been and will be so successful can be found in two simple words: sustanaibility and manageability. Within the optics of the WCM, a working team face every decision is going to take (maintanance, logistic, quality assurance, safety, management, ...) on the basis of their economical incidence. The team activity is oriented toward the accomplishment of small, focused projects (Kaizen) whose objectives are "zero defects, zero faults, zero incidents, zero remaining stocks": in other words, the final target, which is a consistent reduction of the production plant costs, is ensured by the autonomy of the team itself (that's why decisions are "manageable"), using parameters that are always taking into account the economical aspect (and thus "sustainable").

Speaking about near term technological investments, the current business plan introduces a forecast regarding the electrification of vehicles within the FCA group. In the last years, FCA has displayed skepticism regarding the electric cars \cite{fca_electric_bet}, but for the upcoming years, substantial investments in the sector will allow to introduce the proper technology with the aim to provide FCA a place between the electrical automotive leaders in the world \cite{fca_financial_overview}. Because electrical ignition has several open issues on its side, where battery production and disposal are questionably only two of the most dangerous ecological problem nowadays, FCA has preferred a more conservative approach towards such innovation. The current marketing strategy \cite{fca_financial_overview} seems to dictate a phase of analysis and careful observation, where the company itself is both undertaking plenty of research and at the same time managing to interweave strategical partnership (such that with Waymo in recent times \cite{fca_waymo_partnership}) in order to gain control over innovations that have already gained an acceptable level of maturity.

\subsection{Safety \& Security}

FCA takes safety very seriously, not only by a mere research point of view, but also when it comes down to liability. An outstanding example which justifies the latter assert comes from a recent event, where almost 5 millions vehicles have been subject to recall procedure due to a defect on the cruise control subsystem \cite{fca_5million_recall}. Although the issue can be considered of minor impact (since it affects only vehicles with more than 200'000 km, and the problem itself can occur in 'extremely rare circumstances', as stated by FCA itself), FCA opted to a full recall, even if it would imply significant expenses. Engineering issues can always happen, but these are the circumstances that highlight the reliability of a company, and where a quality service comes into play.
Safety, security, and in general terms, reliability are such primary concerns of FCA that a constant research is fullfilled in order to provide the bleeding-edge technology against threats or damages. Notorius recent examples are the Rear Cross Traffic Assist (RCTS) \cite{fca_rear_cross_traffic_assist}, which is helpful in blind driveways, the ArcelorMittal's latest Fortiform® alloy used in vehicles floors, which drastically reduces the costs, but also ensures higher torsion performances and reduces consumptions, due to the overall more lightweight structure \cite{fca_fortiform}.
Is also worth consider that safety and security are also tested by standard procedures, like crash tests, and lately FCA has exceedeed the 20'000 crash test from 1961 to today \cite{fca_20thousand_crash_tests}. This milestone represents more than 50 years of activity where FCA has joined a diverse range of technical skills and standardized methods in order to achieve safety level thant were once unthinkable. The excellence of results is also confirmed by the numerous quality awards conferred by independent third-party organizations, such as the Chrysler Pacifica recognition as "Top Safety Pick" by IIHS \cite{fca_pacifica_iihs}, which is one of the latest and most significant achievements.


\subsection{Self-driving technologies}

FCA started a strategic partnership with the self-driving car service Waymo, belonging under the Alphabet umbrella, since May 2016. The CEO at the time Sergio Marchionne commented: “Strategic partnerships, such as the one we have with Waymo, will help to drive innovative technology to the forefront. [...] FCA is committed to bringing self-driving technology to our customers in a manner that is safe, efficient and realistic”.
Those words entirely reflect the approach that FCA maintains to bring fresh new, qualitative technology into the automotive market.
A first achievement in this sector has been the production of 62,000 Chrysler hybrid minivans to be deploy over the self-driving car service Waymo (FCAW); moreover, the Waymo partnership is not the sole one: in 2016, an agreement between FCA and Google establishes a cooperation to develop a new generation of autonomous vehicles
\cite{fca_google} that will be ready for the market by the end of the decade. These collaborations are clearly representatives of the company's serious intentions to become a leader in this rather new driving paradigm \cite{fca_waymo_partnership}. With these cooperation, FCA aims to develop new forms of technologies that will enhance self-driving management tools, but also to acquire knowledge of possible new unexpected applications where autonomous vehicles could be suddenly employed.
It's worth noting that the self-driving phenomena has not only technological concerns, but also social ones: since nowadays cars are parked 95\% of the time \cite{fca_cars_95_parked}, self-driving cars could be a better option compared to conventional cars, meaning that a fewer number of cars can fulfill the requirements of an higher amount of customers. It has to be considered the reduced stress factor too, since driver would not be exposed to the parking stress attitude, nowadays so commonly spread out in crowded commercial zones. The results can be easily drawn: even if less cars will be built and sold, companies can raise the price of the single unit in order to guarantee more quality and way more tested features. Fewer cars and more quality would means less pollution and much more satisfied customers, so, a win-win for both the producer and the consumer categories. 
Although this situation could be considered a radical change, Waymo is one of the few that is prone to accept the risks, due to the experimental nature of habits conversion itself; FCA is aware of such controversial aspects too, but, nonetheless, is still ready to welcome every possible outcome as Waymo is, since the collaboration keeps going on strongly, having it been recently renewed\cite{fca_waymo_renew}.

\subsection{Environmental Sustainability}
FCA has always been aware of the fact that petrol engines cars are one of main cause of environmental pollution, between many others. Being the ignition mechanisms the root of the problem itself, it's clear that it represents the starting point for a research of more ecologically sustainable solutions.
The first pivotal result came with, LPG gas based engines, almost 30 years ago, and since the launch of the first LPG vehicle, FCA is still one of the biggest producer of natural gas vehicles in the world \cite{FCA_sustainability}.
Nowadays LPG gas usage is considered more a short term solution, rather than a definitive one, and there are many reasons behind such perspective. The rationale is than LPG gas is not a renewable resource and, although not harmful as conventional petrol -as a combustible-, is still a not-towards-zero pollution method of ignition. As a result, FCA is experimenting different approaches, where the most significant one are electrical engines; although electricity can still questionably be considered not mature enough for a mass scale adoption, FCA has undergone some attempts to reach the market, as showed by the launch of five new completely electric Maserati models \cite{Maserati_electric}. The reasons behind considering electricity a final solution to the automotive pollution problem is underlying the fact that most of it is produced from non-renewable sources, which virtually doesn't add any benefit from an ecological standpoint.
Another equally critical aspect is that environmental pollution problem doesn't involve only the automotive companies, but also the customers: for instance, the three most sold vehicles in the US are all pickup trucks \cite{US_best-selling-vehicle}. This is due to the relatively low price of the vehicles in this category, and the almost unbeatable price of the required gasoline. These aspects, together with reliability and versatility concerns, are making customers much more prone to buy a pick-up rather than a smaller, greener car.
Given the complex nature of the problem, which intuitively has many fading shadows under its amplitude, FCA is trying to properly tackle the necessity of a sustainable transportation still pushing LPG technology for the upcoming years, but evaluating a diversified range of alternatives, heavily investing on the research of solutions which can fulfill the purpose, also if they appear immature at the present moment.
Conducting a successful business is not a mere process of introducing innovative ideas and technologies, but also to spot the right time when a certain solution has reached enough maturity to be effectively posed on the market. Exploiting this adequate level of maturity, consensus can be gained through mediatic momentum and therefore revenues will be automatically triggered. FCA knows it, and that's the actual motivation behind the conservative position about electric vehicles and their spreading as today.

\subsection{Maintenance}

A good brand does not only sells a quality product, but also has care of the services to provide in order to let the product itself gain an additional value.
In the automotive world, maintenance is often a crucial part of the whole picture, much more than many other sectors. FCA has a unique approach to maintenance services, bringing a centralized organism whose name is "Mopar". As already said, Mopar stands for "MOtor PARts", and has more than 80 years of history behind his back. Mopar provides assistance and replaceable parts for every FCA vehicle, and the only, specific commitment that this organism has is to furnish components that fit perfectly, guaranteeing availability, reliability and fast response times.
Having a unified entity that takes care of the plethora of maintenance services has several positive outcomes. First of all, knowledge and expertise can be managed in a more straightforward way, leading to working teams more homogeneously distributed in competences, and overall, able to inter-cooperate easily. Second, this choice fully integrates with the main whole business model that FCA conducts through the WCM, thus showing the benefits that the latter provides in the most effective way. Third, the marketing and technical sectors within Mopar can communicate with more practical and collaborative methods, since Mopar acts as it would be a detached company, managing its own personnel and working for its own profit \cite{maintenance_mopar}.
In our opinion, given these concerns, Mopar has few competitors that can reach the same professionality level in its own realm.